We implemented \textit{ComplexOvercooked} based on Gym module with high scalability. The underlying logic and interfaces of the game are implemented using the Pygame module. In this section, we provide a brief introduction of some key classes of the environment and examples to start the game and to modify the configurable settings.\\
\textbf{The RL environment class}  \ In the \texttt{OvercookPygameEnv} class, two key methods stand out:  \texttt{reset()} for initiating a new episode, and \texttt{step()} for feeding the agent's action into the model at each timestep. Both return a quadruple of corresponding environment information:
\begin{itemize}
\item \texttt{nobs}: a tuple corresponding to the number of agents, each element being an array containing the agent's observation in the environment.
\item \texttt{rewards}: a tuple also matching the number of agents, where each element represents the reward obtained by the agent in the current environment.
\item \texttt{dones}: a boolean variable that determines whether the current environment timestep has reached completion.
\item \texttt{infos}: a dictionary that stores various useful pieces of information about the current environment. Within \texttt{infos}, several details are stored that, in our design, can aid in the iteration of algorithms to accelerate them. 
\item \texttt{shaped\_r} denotes the additional shaped reward obtained by an agent in the current environment. 

% \item \texttt{tasksequence} describes a list of events occurring to an agent in the environment, composed of both shaped reward events and sparse reward events, recorded in a natural language format. We hope this information will be useful for Language Models (LLMs) in understanding the environment.
% \item \texttt{aviactions} is a tuple the length of the agents, where each element represents all the legal actions available to that agent at the moment.
\end{itemize}
An example of using our environment is shown in Listing 1.
\begin{lstlisting}[language=Python, caption=Python example to start the game engine, label=code:example1]
from overcook_pygame.overcook_gym_env import OvercookPygameEnv

# load overcook pygame environment
env = OvercookPygameEnv(map_name='supereasy', ifrender=True, debug=True)
nobs, share_obs, available_actions = env.reset()
done = False

# start an episode
while not done:
    random_action = np.random.randint(0, 6, size=env.n_agents)
    
    nobs, share_obs, rewards, dones, infos, available_actions = env.step(random_action)
    
    done = dones[0]
\end{lstlisting}\\
\textbf{Personalized configuration} \ The game layout, number of agents, and order configurations can be achieved by editing a configuration file as shown below. Here, \texttt{name} refers to the name of the layout; \texttt{layout} is an array representing the specific layout configuration, where \texttt{X}, \texttt{B}, \texttt{F}, \texttt{D}, \texttt{L}, \texttt{U}, \texttt{E}, \texttt{C} denote the empty table, the trash bin, the fish dispenser, the dish dispenser, the lemon dispenser, the cutting board, the serving area and the pot, respectively. Arabic numerals represent players identities.\texttt{items} represent all possible recipes in the environment; \texttt{task} represents the set of possible orders and their corresponding rewards; and \texttt{tasknum} specifies the number of orders that appear simultaneously. By editing this file, the complexity of the game can be customized.

\begin{lstlisting}[language=JSON, caption=maps.json example of the env configurations=code:example2]
"4playereasy":{
    "name": "4playereasy",
    "layout": [
         "XBXMEX__",
         "C1__4U__",
         "X____X__",
         "H3__2T__",
         "XFXDLX__",
         "________"],

    "items": ["cookedfish","ACtomato","BCtomato","cookedbeef","rawbeef","hamburger", "ACtomatocookedbeefhamburger","ACtomatohamburger","cookedbeefhamburger", "dish", "BClemon", "AClemon", "rawfish", "AClemoncookedfish"],
    "task": {"AClemoncookedfish":20, "cookedfish":10, "ACtomatocookedbeefhamburger":25,"cookedbeefhamburger":15},
    "tasknum": 2,
    "players": 4
      }
\end{lstlisting}